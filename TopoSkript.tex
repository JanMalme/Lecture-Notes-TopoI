\documentclass{TemplateLecture}

\renewcommand{\LectureName}{Topology I}
\renewcommand{\ProfName}{Stefan Schwede}
\renewcommand{\Semester}{WiSe 2024/25}
\renewcommand{\mName}{Jan Malmström}

\begin{document}

\pagenumbering{arabic}
\section{CW-Complexes}

The name abbreviates compact-Closure-Weak-Topology. They are \enquote{nice} classes of spaces for the purpose of homotopy theory/algebraic topology. They are build by successively attaching cells.

The \(n\)-cell is \(D^n = \set{x \in \IR^n: \abs{x} \leq 1}\). It may also be called \(n\)-balls or \(n\)-discs.
\(S^{n-1} = \partial D^n = \set{x\in \IR^n : \abs{x} = 1}\) is the \(n-1\)-Sphere.
See figure \ref{fig:exDn} for examples.

\begin{figure}
    \centering
    \includegraphics[width=0.8\linewidth]{pic/exDn.png}
    \caption{\(D^n\) and \(S^{n-1}\) for small \(n\)}
    \label{fig:exDn}
\end{figure}


\subsection{Definition}

\begin{construction}
    Let \(n \geq 0\), let \(f\colon S^{n-1} \to X\) be a continuous map, the \emph{attaching map}\index{Attaching map}. We form the quotient space
    \[X \cup_{f, \partial D^n} D^n = X \cup_f D^n = X \cup_{\partial D^n} D^n \coloneq X \amalg D^n/\sim\]
    where \(\sim\) is the equivalencce relation on \(X\amalg D^n\) genearted by \(\forall \; x \in S^{n-1}: f(x) \sim x\).
\end{construction}

\textbf{Terminology.} We say: \enquote{\(X\cup_f D^n\) is obtained by attaching an \(n\)-cell to \(X\) along \(f\)}.

\begin{bsp}
    \begin{itemize}
        \item \(X\cup_f D^0 = X \amalg D^0\)
        \item \(\set{*} \cup_{S^{n-1}} D^n = D^n/ \sim = D^n/ S^{n-1} \cong S^n\)
        
        In this example \(\sim\) identifies all of \(S^{n-1}\) to a point, which then is homeomorphic to \(S^n\)\footnote{supposed as known}.
        
        \item Remark, that the attaching map matters greatly. See figure \ref{fig:exAt}
        \[S^{n-1} \cup_f D^n \cong D^n \quad \text{with } f = \Id\colon S^{n-1} \to S^{n-1}\]
        \[S^{n-1} \cup_f D^n \quad \text{with } f \colon S^{n-1} \to S^{n-1} \text{ constant}\]
        \begin{figure}
            \centering
            \includegraphics[width=0.8\linewidth]{pic/exAt.png}
            \caption{The attaching map influences how \(D^n\) is attached.}
            \label{fig:exAt}
        \end{figure}
    \end{itemize}
\end{bsp}

\subsubsection*{Simultaneous attachment of several cells}

Let \(J\) be an indexing\footnote{\enquote{indexing}does not carry mathematical meaning} set, considered as a discrete space (\(J = \emptyset\) is allowed).

Give \(J \times D^n\) the product topology, then
\[J \times D^n \cong \coprod_{j \in J} \set{j} \times D^n\text{\footnote{supposed as known}}\]
as a topological space. The \(\coprod\) represents the disjoint union topology.

It follows, that
\[\begin{tikzcd}
    \set{\text{continuous maps } f\colon J \times D^n \to X} \ar[d, phantom, "\cong", sloped] & f \ar[d, mapsto] \\
    \set{\text{J-indexed families of continuous maps } \set{f_j \colon D^n \to X}_{j \in J}} & f_j = f(j, \_)
\end{tikzcd}\]
We will identify them from now on.

\begin{defi}{}{}
    Let \(f \colon J\times \partial D^n \to X\) be a continuous map, the \emph{attaching map}\index{Attaching map}.
    \[X\cup_{f, J\times \partial D^n} J\times D^n = X \cup_f J\times D^n = X \cup_{J\times \partial D^n} J \times D^n \coloneq X \amalg J\times D^n /\sim\]
    where \(\sim\) is the equivalence relation generated by \(f(x) \sim x\) for all \(x \in J\times \partial D^n\).
\end{defi}



\textbf{Remark.} Write
\[p\colon X \amalg J\times D^n \to X\cup_f J\times D^n\]
for the quotient map. From the universal property of the qoutient map follows: Given maps \(g \colon X \to Y\) and \(\Psi_j \colon D^n \to Y\) such that \(g(f_j(x)) = \psi_j(x)\) for all \(j \in J, x \in \partial D^n\) there is a unique map \(\psi\colon X\cup_f J\times D^n \to Y\), such that
\[\psi \circ p = g + \coprod_{j \in J} \psi_j \colon X \amalg (J\times D^n) \to Y\]
and \(\psi\) is continuous iff \(g\) and all \(f_j\) are continuous.

Remeber the quotient-topology: A subset \(O\) in \(X \cup_f J\times D^n\) is open iff \(p^{-1}(O)\) is open in \(X \amalg J\times D^n\). This is equivalent to \(p^{-1}(O) \cap X\) is open in \(X\) and for all \(j \in J\) \(p^{-1}(O) \cap j \times D^n\) is open in \(D^n\).


\(X\) is a closed subspace of \(X\cup_f J\times D^n\) \(J \times \mathring{D^n}\) is an open subset of \(X\cup_f J\times D^n\)
\(X \cup_f J\times D^n\) is as a set (not as a space) the disjoint union of \(X\) and \(J\times \mathring{D^n}\).
We elaborate

\begin{proposition}
    \begin{enumerate}
        \item The composition
        \[\begin{tikzcd}
            X \ar[r] & X\amalg (J\times D^n) \ar[r, "p"] & X\cup_f J\times D^n
        \end{tikzcd}\]
        is a closed embedding (i.e. a closed injective map).
        \item The composition 
        \[\begin{tikzcd}
            J\times \mathring{D^n} \ar[r, hook, "incl"] & J\times D^n \ar[r] & X\amalg J\times D^n \ar[r, "p"] & X \cup_f J \times D^n
        \end{tikzcd}\]
        is an open embedding (i.e. injective and open)
        \item The underlying set of \(X\cup_f J\times D^n\) is the disjoint union of the image of \(X\) and \(J\times \mathring{D^n}\).
    \end{enumerate}
\end{proposition}

\begin{proof}
    Suppose \(M \subseteq X \amalg J\times D^n\) is saturated, i.e. \(M = p^{-1}(p(M))\). If \(M\) is saturated and open, then \(p(M)\) is open in \(X\cup_f J\times D^n\).
    \begin{enumerate}
        \item \begin{description}
            \item[\(n = 0\)] \(X\cup J\times D^0 = X \amalg J\times D^0\) is obvious.
            \item[\(n \geq 1\)] let \(r\colon D^n \to S^{n-1}\) be a map, such that \(r(x) = x\) for all \(x \in S^{n-1}\).This \underline{cannot} be done continuously.
            Define \(X \amalg J\times D^n \to X\) by \(x \mapsto x, (j,y) \mapsto r(y)\). This is compatible with the equivalence relation, so it descends to a (noncontinuous) map \(X \cup_f J\times D^n \to X\). This prooves injectivity.
            To show this is a closed map, we consider a closed subset \(A \subseteq X\). Then
            \(p^{-1}(p(A)) = A\amalg f^{-1}(A) \subseteq X\amalg J\times D^n\)
            \(\subset J\times \partial D^n \subset J\times D^n\) is closed in \(X \amalg J\times D^n\).
            So \(p(A)\) is closed in \(X\cup_f J\times D^n\).
        \end{description}
        \item All points in \(J \times \mathring{D^n}\) are their own equivalence classes, so the map is injective. To show that the map of 2. is open, we let \(B\) be an open subset of \(J\times \mathring{D^n}\). This is then also open in \(J \times D^n\). \(p^{-1}(p(B)) = \emptyset \amalg B\subset X\amalg J\times D^n\) open, so \(p(B)\) is open in \(X\cup_f J\times D^n\).
        
        \item I think this was prooven with a picture I didn't draw.
    \end{enumerate}
\end{proof}

\textbf{Exercise.} Let \(V_j\) be an open subset of \(D^n\) for every \(j \in J\), such that \(V_j \supset \partial D^n\). Show, that the set \(V = X \cup \bigcup_{j \in J} V_j\) is open in \(X\cup_f J\times D^n\).

From now on we often identify \(X\) with its image in \(X \cup_f J\times D^n\) and \(J\times \mathring{D^n}\) with its image in \(X \cup_f J\times D^n\)

\begin{defi}{Compactness}{}
    A space \(X\) is \emph{compact}, if it is Hausdorff (any two points can be separated by two disjoint open sets) and \emph{quasicompact} (any open cover has a finite subcover).
\end{defi}

\textbf{Remark.} Some literature defines compactness equivalent to quasicompactness. This lecture uses the definition that was given.

\begin{thm}{}{comp}
    Let \(f\colon J \times \partial D^n \to X\) be a continuous attaching map.
    \begin{itemize}
        \item If \(X\) is Hausdorff, then so is \(X \cup_f J\times D^n\).
        \item If \(X\) is compact and \(J\) is finite, then \(X \cup_f J\times D^n\) is compact.
        \item Let \(K\) be a quasicompact subset of \(X\cup_f J\times D^n.\) Then \(K\cap (\set{j} \times \mathring{D^n}) = \emptyset\) for almost all\footnote{mathematical term for all but finitely many.} \(j \in J\).
    \end{itemize}
\end{thm}

\begin{lem}{}{}
    There exists an open neighborhood \(V\) of \(X\) in \(X\cup_f J\times D^n\) and a continuous map \(r\colon V\to X\) that is the identity on \(X\). (\(X\) is a neighborhood retract inside \(X \cup_f J\times D^n\)).
\end{lem}

\begin{proof}
    \begin{figure}
        \centering
        \includegraphics[width=0.8\linewidth]{pic/exRt.png}
        \caption{If a point in \(D^n\) is missing, it can be continuously retracted.}
        \label{fig:exRt}
    \end{figure}
    See figure \ref{fig:exRt}. We take
    \(V = X\cup_{J\times \partial D^n} J\times (D^n\setminus {0})\). This is open in \(X \cup_f J\times D^n\). We define \(r \colon V\to X\) by \(x\mapsto x, (j,z) \mapsto f(j, z/\abs{z})\).
\end{proof}

\begin{proof}[Proof of theorem \ref{thm:comp}]\leavevmode
    \begin{enumerate}
        \item \begin{description}
            \item[Case 1] \(x,y \in J\times \mathring{D^n}\). Since \(\mathring{D^n}\) is Hausdorff, so is \(J\times \mathring{D^n}\), so we can separate \(x\) and \(y\) by open disjoint subsets in \(J\times \mathring{D^n}\), Since \(J\times \mathring{D^n}\) is open in \(X \cup_f J\times D^n\), theses subsets are also open in \(X\cup_f J\times D^n\).
            \item[Case 2] \(x\in X, y \in \set{j} \times \mathring{D^n}\). We choose an \(y \in O_y \subset j \times D^n\) open \(j \times \partial D^n \subseteq V_j \subseteq j\times D^n\) s.t. \(O_j \cap V_j = \emptyset\).
            Then \(V\coloneq X\cup V_j \cup \bigcup_{k \in J\setminus\set{j}} D^n\) is open\footnote{by an exercise.} in \(X \cup_f J \times D^n\). \(V \cap O_j = \emptyset\), \(x \in V, y \in O_j\).
            \item[Case 3] \(x,y \in X\). Since \(X\) is Hausdorff, there are open subsets \(O_x, O_y\) of \(X\) with \(x \in O_x\), \(y \in O_y\), \(O_x \cap O_y = \emptyset\).
            We let \(V\) be an open subset of \(X \cup_f J \times D^n\) with a continuous retraction \(r\colon V\to X\), \(r\rvert_X = \Id_X\). Then \(x \in r^{-1}(O_x), y \in r^{-1}(O_y)\), \(r^{-1}(O_y), r^{-1}(O_y)\) are open, and disjoint.
        \end{description}
        \item If \(X\) is compact and \(J\) is finite, then \(X\amalg J\times D^n = X \amalg \coprod_{j \in J} \set{j} \times D^n\) is compact hence also the quotient space \(X \cup_f J\times D^n\) is quasi-compact. Hausdorff is inherited by \(1.\).
        \item Let \(K\) be a quasicompact subset of \(X \cup_{J\times \inner{D^n}} J\times D^n\). We define subsets \(V_j\) of \(D^n\) for all \(j \in J\) as follows: If \(K \cap (j \times \mathring{D^n}) = \emptyset\), we set \(V_j = D^n\). If \(K \cap (j\times \mathring{D^n}) \neq \emptyset\), we choose a \(V_j\), that doen't contain at least one point of \(K\), is open, and contains \(\partial D^n\). Now
        \[(X\bigcup_{j \in J} V_j) \cup \bigcup_{j \in J} \set{j} \times \mathring{D^n}\]
        is an open cover of \(X\cup_f J\times D^n\). Since \(K\) is quasicompact, there is a finite subset \(L\) of \(J\) such that
        \[K \subset (X\cup_{j \in J} V_j) \cup \bigcup_{j \in L} \set{j} \times \mathring{D^n}.\]
    \end{enumerate}
\end{proof}

\begin{bsp}[Hawaiian Earrings]
    The set
    \[H = H_1 \cup H_2 \cup H_3 \cup \dots = \bigcup_{i \geq 1} H_i\]
    wherein \(H_i\) is the circle in \(\IR^2\) with radius \(1/i\) and center \((1/i, 0)\), equipped with the subspace topology of \(\IR^2\) is called the Hawaiin earrings (see figure \ref{fig:exHw}).
    \begin{figure}
        \centering
        \includegraphics[width=0.5\linewidth]{pic/exHw.png}
        \caption{Hawaiian earrings}
        \label{fig:exHw}
    \end{figure}

    Is \(H\) obtained from \(\set{(0,0)}\) by attaching countably many 1-cells? It is not.

    Consider a continuous map \(\psi_j\colon D^l = [-1,1]\) such that it is a surjective, and \([-1,1]/-1\sim 1\) onto  \(H_j \subset H\) is a homeomorphism.

    \[\set{(0,0)} \amalg \IN \times D^1 \to H, \quad (j,x) \mapsto \psi_j(x)\]
    is a continuous surjection. Then
    \[\set{(0,0)} \cup_{\IN\times \partial D^1} \IN \times D^1 \to H\]
    is a continuous bijection. However, it is not a homeomorphism.

    Consider \(V = \set{(0,0)} \cup_{\IN \times \partial D^1} \IN \times ([-1,0) \cup (0,1])\). This is open in \(\set{(0,0)} \cup_{\IN\times \partial D^1} \IN \times D^1\). Its complement is closed, but the image of that complement, \((1/n,0)_{n\in \IN}\) is not closed in \(H\).
\end{bsp}

\begin{defi}{CW-Complex}{cwcomplex}
    A relative \emph{CW-complex}\index{CW-} is a space \(X\) equipped with a sequence of closed subspaces
    \[A = X_{-1} \subseteq X_0 \subseteq X_1 \subseteq \dots \subseteq X_n \subseteq \dots \]
    such that
    \begin{enumerate}
        \item For every \(n \geq 0\) \(X_n\) can be obtained from \(X_{n-1}\) by attaching \(n\)-cells.
        \item \(X = \bigcup_{n \geq 0} X_n\) and \(X\) has the weak topology with respect to the sequences.
    \end{enumerate}

    precisely:
    \begin{enumerate}
        \item There exists an index set \(J\), a continuous map \(f \colon J \times \partial D^n \to X_{n-1}\) and a homeomorphism \(\psi\colon X_{n-1} \cup_f J\times D^n \to X_n\) that is the identity on \(X_{n-1}\).
        \item A subset \(O\) of \(X\) is open in \(X\) iff \(O\cap X_n\) is open in \(X_n\) for all \(n \geq 0\).
    \end{enumerate}
\end{defi}

    \textbf{Remark.} 2. is equivalent to: a subset \(C\) of \(X\) is closed in \(X\) iff \(C\cap X_n\) is closed in \(X_n\) for all \(n \geq 0\).

    2. implies, that a map \(f\colon X\to Y\) is already continuous if \(f\rvert_{X_n}\colon X_n \to Y\) is continuous for all \(n \geq 0\).

\begin{notation}
    \begin{itemize}
        \item We usually say \((X,A)\) is a relative CW-complex and leave the \(X_n\) implicit.
        \item For \(A = \emptyset\), \(X\) is called a absolute CW-complex, or just a CW-complex.
        \item The subspace \(X_n\) in a CW-complex is the \(n\)-skeleton.
        \item A relative CW-complex \((X,A)\) is finite-dimensional if \(X_n = X\) for some \(n \geq 0\).
        \item A relative CW-complex \((X,A)\) is finite, if there are only finitely many cells altogether.
        \item Once chosen a homeomorphism \(\psi\) as above, then the characteristic map  of the \(j\)-th \(n\)-cell \(\chi_j\) is the composite
        \[\begin{tikzcd}
            D^n \ar[r, "{(j, \_)}"] & X_{n-1}\cup_{J\times \partial D^n} J \times D^n \ar[r, "\psi", "\cong"'] & X_n \ar[r, hook] & X
        \end{tikzcd}\]
        \(\chi_j\rvert_{\mathring{D^n}} \colon \inner{D^n} \to \chi_j(\mathring{D^n})\) is a homeomorphism onto its image, which is one path component of \(X_n \setminus X_{n-1}\). The restriction
        \[f_j \coloneq \chi_j\rvert_{\partial D^n}\colon \partial D^n \to X_{n-1}\]
        is called the attaching map as before.
    \end{itemize}
\end{notation}

\textbf{Comment.} The space \(X_n \setminus X_{n-1}\) is a disjoint union of open cells \(\mathring{D^n}\). So the indexing set could be taken as \(\pi_0(X_n\setminus X_{n-1})\). Esspecially its cardinality is fixed.

It can be shown, that for every path-component of \(X_n \setminus X_{n-1}\) there exists a homeomorphism
\[f \colon \mathring{D^n} \to \text{ that path-component}\]
that extends to a continuous map \(\bar f \colon D^n \to X_n\).


\begin{example}
    \begin{itemize}
        \item Any discrete space is an absolute \(0\)-dimensional CW-complex.
        \item Let \(z \in S^n\) be any point. Then  the minimal CW-structure on \(S^n\) is
        \[X_{-1} = \emptyset, \quad X_0 = \set{z} = X_1 = \dots = X_{n-1}\]
        \[S^n = X_n = X_{n+1} = \dots\]
        It consists of one \(0\)-cell and one \(n\)-cell. 
        This can be seen, because \(S^n \cong D^n / \partial D^{n-1}\) by \(\partial D^{n-1} \to \set{z}\).

        The CW-strucuture on a given space \(X\) is not unique. For example a different CW-structure on \(S^2\) consists of two of each \(0,1\) and \(2\)-cells. See figure \ref{fig:exS2} for the construction.
    \begin{figure}
        \centering
        \includegraphics[width=0.5\linewidth]{pic/exS2.png}
        \caption{\(S^2\) is built from several cells.}
        \label{fig:exS2}
    \end{figure}
        Analog, \(S^n\) is a CW-complex with \(2\) \(i\)-cells for \(i = 0, \dots, n\).

        Also a CW-structure: For \(S^1\) pick any finite subset \(A \subseteq S^1\). Then \(S^1\) has a CW-structure with \(X_{-1} = \emptyset, X_1 = A, X_2 = S^1\). n 0 cells n 1 cells.

        It can be shown, that any non-discrete space, that admits an absolute CW-structure admits uncountably many different CW-structures.

        \item[\textbf{Preview.}] The Euler characteristic of a finite absolute CW-complex is
        \[\chi(X) = \sum_{n \geq 0} (-1)^n \#n\text{-cells}\]
        does not depend on the CW-structure. We will eventually show this using singular homology. 
    \end{itemize}
\end{example}

\begin{thm}{}{}
    Let \((X,A)\) be a relative CW-complex.
    \begin{enumerate}
        \item If \(A\) is Hausdorff, then so is \(X\).
        \item If \(A\) is compact and \((X,A)\) is finite, then \(X\) is also compact.
    \end{enumerate}
\end{thm}

\begin{proof}
    Because \(X_{-1} = A\) is Hausdorff and \(X_n\) can be obtained from \(X_{n-1}\), by attaching cells, inductively \(X_n\) is Hausdorff for all \(n \geq 0\).

    \textbf{Claim.} Let \(O_n, P_n\) be open disjoint subsets of \(X_n\). Then there exist disjoint open subsets \(O_{n+1}, P_{n+1}\) of \(X_{n+1}\), such that \(O_n = O_{n+1} \cap X_n, P_n = P_{n+1} \cap X_n\).
    \begin{proof}
        Since \(X_{n+1}\) can be obtained from \(X_n\) by attaching \((n+1)\)-cells \(X_n\) is a neighborhood retract in \(X_{n+1}\), i.e. there is a open neighborhood \(V\) of \(X_n\) in \(X_{n+1}\) and a continuous retraction \(r\colon V\to X_n\) with \(r \rvert_{X_n} = \Id\). We set \(O_{n+1} = r^{-1}(O_n), P_{n+1} = r^{-1}(P_n)\).
    \end{proof}
    We proove the Hausdorff property: Let \(x,y \in X\) be disjoint points. Since \(X = \bigcup_{n \in \IN} X_n\). Then for some \(n \geq 0\), \(x,y \in X_n\). Since \(X_n\) is Hausdorff, there are open, disjoint subsets \(O_n, P_n\) of \(X_n\) with \(x \in O_n, y \in P_n\). Inductiveley use the claim to find open disjoint subsets \(O_m\), \(P_m\) of \(X_m\) for all \(m \geq n\), such that \(O_{m+1} \cap X_m = O_m, P_{m+1} \cap X_m = O_m\) for all \(m \geq n\). Then set \(O = \bigcup_{m \geq n} O_m, P = \bigcup_{m\geq n} PM\) disjoint subsets of \(X\) and open in \(X\) by the weak topology, as \(O\cap X_m = O_m\) open in \(X_m\).

    For compactness, Induction over \(n\), such that \(X_n\) is compact because \(X_n\) is obtained from \(X_{n-1}\) by attaching finitely many cells. Also \(X = X_n\) for sufficently large \(n\). So \(X\) is compact.
\end{proof}

\textbf{Note.} Suppose that \(X\) admits a CW-structure. Then the following are equivalent: \(X\) admits a finite CW-structure \(\Lra\) \(X\) is compact.

From now on we assume, the base \(A\) in a relative CW-complex \(X,A\) is Hausdorff. Then \(X\) is also Hausdorff.

\begin{thm}{}{}
    Let \(X,A\) be a relative CW-complex.
    \begin{enumerate}
        \item The closure of every open \(n\)-cell (\(=\) path component of \(X_n\setminus X_{n-1}\)) is compact.
        \item Let \(\chi\colon D^n \to X\) be a characteristic map for some \(n\)-cell, then  the image \(\chi(D^n)\) is the closure of the open cell \(\chi(\mathring{D^n})\)
        \item Let \(U\) be a subset of \(X\) s.t. \(A\subseteq U\). Suppose that the intersection of \(U\) with the closure of every cell is closed. Then \(U\) is closed in \(X\).
    \end{enumerate}
\end{thm}

\textbf{Warning.} The closure of a cell is not necessary a closed cell.
See for example the minimal CW-tructure on \(S^2\). The closure of the open 2-cell \(S^2\setminus \set{z}\) is \(S^2 \neq D^2\).

\begin{proof}\leavevmode
    \begin{enumerate}
        \item By definition every open \(n\)-cells admits a characteristic map \(\chi\colon D^n \to X_n\) continuous s.t. \(\chi\rvert_{\mathring{D^n}}\) is a homeomorphis onto the open cell. Then
        \[\chi(D^n) \subseteq \text{ closure of } \chi(\mathring{D^n})\]
        and as \(D^n\) is compact, and \(X\) is Hausdorff, \(chi(D^n)\) is closed, so \(\chi(D^n) = \text{ closure of } \chi(\inner{D^n})\). As \(D^n\) is compact, this is also.

        \item Already contained in 1.
        
        \item Let \(U\subseteq X\) be as in \(2\). It suffices to show that \(U\cap X_n\) is closed in \(X_n\) for all \(n \geq 0\) (weak topology). We argue by induction on \(n\).
        \begin{description}
            \item[\(n = -1\)] \(U\cap X_{-1} = U\cap A = A\) closed in \(A = X_{-1}\).
            \item[\(n \geq 0\)]  We choose a homeomorphism \(\psi\colon X_n \cong X_{n-1} \cup_{J\times \partial D^n} J\times D^n\) that is the identity on \(X_{n-1}\). We let
            \[p\colon X_{n-1} \amalg J\times D^n \to X_{n-1} \cup_{J\times \partial D^n} J\times D^n \cong X_n\]
            be the quotient map. Then
            \[p^{-1}(U\cap X_n) = \underbrace{(U\cap X_{n-1})}_{\text{closed by induction}} \amalg \coprod_{j \in J} p^{-1}\underbrace{(U\cap \text{ closure of j-th n-cell})}_{\text{closed by hypothesis}}\]
            This is closed as a subspace of \( X_{n-1} \amalg J\times D^n\) and hence \(U\cap X_n\) is closed in \(X_n\)
        \end{description}
    \end{enumerate}
\end{proof}

\subsection{CW-subcomplexes}

\begin{proposition}
    Let \(A\) be a Hausdorff-space, \(X = A \cup_f J\times D^n\) obtained from \(A\) by attaching \(n\)-cells. Let \(Y \subseteq X\) be a subspace, such that
    \begin{itemize}
        \item \(Y \cap A\) is closed in \(A\)
        \item \(Y\) can be obtained from \(A \cap Y\) by attaching \(n\)-cells.
        \item \(Y \cap (J\times \inner{D^n})\) is a union of path components of \(J\times \inner{D^n}\).
    \end{itemize}
    Then \(Y\) is closed in \(X\).
\end{proposition}

\begin{proof}
    \textbf{Claim.} If \(Y \cap \set{j} \times \inner{D^n} \neq \emptyset\) (\(\Lra\) \(j\times \inner{D^n} \subseteq Y\)).
    Then \(Y\) contains the closure of \(j\times \inner{D^n}\) in \(X\). ( = the closure of this cell).
    \begin{proof}
        \(Y\) can be obtained from \(Y \cap A\) by attaching \(n\)-cells and \(Y\setminus (Y\cap A)\) is a union of some of the open cells of \(X\setminus A = J\times \inner{D^n}\). Let \(\chi \colon D^n \to Y\) be a characteristic map for the attaching of the \(j\)-th \(n\)-cell to \(Y\).
        \(\chi(\inner{D^n}) = j \times \inner{D^n}\). Since \(D^n\) is compact, \(f(D^n)\) is quasicompact, and hence closed in \(X\) since \(X\) is Hausdorff. Then
        \[j\times \inner{D^n} = \underset{\chi(\inner{D^n})}{\text{closed in } X} \subseteq \chi(D^n) \subseteq Y \subseteq X\]
        and the closure of \(\chi\inner{D^n} = j \times \inner D^n\) is in \(\chi(D^n)\) and hence in \(Y\).  
    \end{proof}
    We let
    \[p\colon A \amalg J\times D^n \to A \cup_f J\times D^n \cong X\]
    be the quotient map. Then 
    \[p^{-1}(Y) = (Y\cap A) \amalg \coprod_{\substack{j \in J \\ Y \cap (j\times \inner{D^n}) \neq \emptyset}} j\times D^n \amalg \coprod_{\substack{j \in J \\ Y\cap (j\times \inner D^n) = \emptyset}} p^{-1}(Y\cap A) \cap (j\times D^n)\]
    So \(Y\) is closed in \(X\).
\end{proof}

\begin{thm}{}{}
    Let \((X,A)\) be a relative CW-complex and \(Y\) a closed subspace of \(X\) with \(A \subseteq Y\). Suppose that for all \(n \geq 0\), \(Y \cap X_n \setminus X_{n-1}\) is a disjoint union of path components of \(X_n \setminus X_{n-1}\). Then \((Y, A)\) is a relative CW-complex with respect to the induced filtration, i.e.
    \[A = Y_{-1} \subseteq Y_0 = (X_0 \cap Y) \subseteq Y_1 = X_1 \cap Y \subseteq \dots \subseteq Y_n = X_n \cap Y \subseteq \dots \]
\end{thm}

\begin{proof}\leavevmode
    \begin{enumerate}
        \item \(Y_n\) can be obtained from \(Y_{n-1}\) by attaching \(n\)-cells.
        Let
        \[I = \set{j \in J\colon Y \cap (j\times\inner {D^n}) \neq \emptyset} = \set{j \in J : j\times \inner{D^n} \subseteq Y}.\]
        Let \(\chi_j \colon D^n \to X_n \subseteq X\) be a charactreistic map for the \(j\)-th \(n\)-cell of \(X\).
        If \(j \in I\), then
        \[\chi(D^n) = \text{closure of } \chi(\inner{D^n})\]
        and since \(Y\) is closed, this is a closed subspace of \(Y\).
        So we can (and will) consider \(\chi\) as a map with target \(Y\cap X_n = Y_n\).
        We get a continuous map
        \[\psi \colon Y_{n-1} \cup_{I\times \partial D^n} I\times D^n \to Y_n\]
        (induced by \(\coprod_{j \in I} \chi_j\)),
        which is bijective because source and target are - as sets - both the disjoint union of \(Y_{n-1}\) and \(I \times \inner{D^n}\). We argue, that \(\psi\) is a closed map and hence a homeomorphism. See
        \[\begin{tikzcd}
            Y_{n-1} \amalg I\times D^n \ar[rr, hook] \ar[d, "q"] & & X_{n-1} \amalg J\times D^n \ar[d, "p"]\\
            Y_{n-1} \cup_{I \times \partial D^n} I\times D^n \ar[r, "\psi"] & Y_n \ar[r, phantom, "\subseteq"] & X_n \\
        \end{tikzcd}\]
        Let \(B\subseteq Y_{n-1} \cup_{I\times \partial D^n} I\times D^n\) be a closed subset, where \(f_j \colon \partial D^n \to X_{n-1}\) is the attaching map for the \(j\)-th \(n\)-cell i.e. \(f_j = \chi_j\rvert_{\partial D^n}\). Then
        \[p^{-1}(\psi(B)) = \underset{q^{-1}(B)}{X_n \amalg I \times D^n} \amalg \coprod_{j \in J\setminus I} j\times f_j^{-1}(B \cap X_{n-1})\]
        With \(f_j = \chi_j \rvert_{\partial D^n} \colon \partial D^n \to X_{n-1}\). As all these are closed, \(p^{-1}(\psi(B))\) is closed. Hence \(\psi(B)\) is closed in \(X_n\) and also in \(Y_n\).

        \item \(Y\) has the weak topology with respect to
        \[Y = Y \cap X = Y \cap (\bigcup_{n \geq 0} X_n) = \bigcup_{n \geq 0} (Y \cap X_n) = \bigcup_{n \geq 0} Y_n.\]
        Let \(B \subseteq Y\) be a subset such that for all \(n \geq 0\), \(B\cap Y_n\) is closed in \(Y_n\). Since \(Y\) is closed in \(X\), \(Y_n\) is closed in \(X_n\), so \(B\cap Y_n\) is closed in \(X_n\). Since \(X\) has the weak topology, \(B\) is closed in \(X\), hence also in \(Y\).
    \end{enumerate}
\end{proof}

\begin{defi}{}{}
    A CW-subcomplex of a relative CW-complex \((X,A)\) is a closed subspace \(Y\) of \(X\), such that \(A \subseteq Y\) and for all \(n \geq 0\) \(Y \cap (X_n\setminus{X_{n-1}})\) is a union of path components of \(X_n\setminus X_{n-1}\).
\end{defi}

\textbf{Note.} Let \((Y,A)\) be a CW-subcomplex of \((X,A)\). Then \((Y,A)\) is a relative CW-complex with respect to the induced filtration.

\begin{thm}{}{}
    Let \((X,A)\) be a relative CW-complex.
    \begin{enumerate}
        \item The closure of every cell is contained in a finite subcomplex.
        \item Every compact subset of \(X\) is contained in a finite subcomplex of \(X\).
    \end{enumerate}
\end{thm}

\textbf{Remark.} The Historically first definition of CW-complexes (J.H.C. Whitehead). A CW-complex is a space X equipped with a decomposition \(X = \dot\bigcup_{n \geq 0, i \in J_n} e_i^n\), such that
\begin{enumerate}
    \item \(e_i^n\) is homeomorphic do \(\inner{D^n}\).
    \item The closure of \(e_i^n\) is contained in the union of finitely many \(e^m_j\)-s (\enquote{closure finite}).
    \item a subset \(Y\) of \(X\) is closed iff \(Y \cap \overline{e_i^n}\) is closed for all \(e_i^n\). then called weak topology.\footnote{The equivalence of this definition to ours will be shown later.}
\end{enumerate}


\begin{proof}
    Since the closure of every cell is compact, \(1\) is a special case of 2.

    Let \(K\) be a compact subset of \(X\).
    \textbf{Claim } There is an \(n \geq 0\), such that \(K \subseteq X_n\).
    \begin{proof}[Proof by contradiction.]
        If \(K \not\subseteq X_n\) for all \(n \geq 0\). Then we can choose points in \(K\) \(x_1, x_2, x_3, \dots \in K\), such that \(x_i \in X_{n_i} \setminus X_{n_i -1}\) for some \(n_1 < n_2 < n_3 < \dots\).
        Set \(D\coloneq \set{x_1, x_2, x_3, \dots}\).

        \textbf{Subclaim.} Every subset of \(D\) is closed in \(X\).
        Let \(S\subseteq D\) be any subset. Thus for all \(n \geq 0\) \(S\cap X_n\) is finite, hence closed in \(X\) (Hausdorff). In particular, \(D\) is
        Closed in \(X\) and contained in \(K\) hence compact.
        But \(D\) has discrete topology and \(D\) is infinite. Contradiction.
    \end{proof}

    Now we assume that the compact subset \(K\) is contained in \(X_n\). We argue by induction over \(n\).
    \begin{description}
        \item[\(n = -1\)] If \(K\) is contained in \(A\), then  \(A,A\) is a finite CW complex.
        \item[\(n \geq 0\)] We choose a representatoin \(X_n \cong X_{n-1} \cup_{J\times \partial D^n} J\times D^n\) We showed earlier, that \(K\) only meets finitely many of the \(n\)-cells in the interior. Set
        \[I = \set{j \in J : K \cap (j\times \inner{D^n}) \neq 0}\]
        a finite subset of \(J\).
        Set
        \[L \coloneq K \cup \bigcup_{j \in I} \underbrace{(\text{ closure of \(j\)-th \(n\)-cell})}_{\text{compact}}\]
        Note that \(L\) is compact.
        Since \(X_{n-1}\) is closed in \(X\), \(L\cap X_{n-1}\) is closed in \(X_{n-1}\), and hence compact. So by induction, \(L\cap X_{n-1}\) is contained in some finite CW-subcomplex of \((X_{n-1},A)\). Then \(K\) is contained in \(Y \cup_{I\times \partial D^n} I\times D^n\), another finite subcomplex of \((X,A)\).
    \end{description}
\end{proof}

\subsection{Cellular approximation theorem}

We will formulate the cellular approximation theorem and spend some time to prove it.

\begin{defi}{}{}
    Let \((X,A)\) and \((Y,B)\) be relative CW-complexes. Let \(f \colon X\to Y\) be a continous map, such that \(f(A) \subseteq B\). The map \(f\) is \emph{cellular}\index{cellular map} if \(f(X_n) \subseteq Y_n\) for all \(n \geq 0\).
\end{defi}

\begin{thm}{Cellular approximation}{CAT}\index{Cellular approximation}
    Let \((X,A)\), \((Y,B)\) be relative CW-complexes, and \(f\colon X \to Y\) continuous with \(f(A) \subseteq B\). Then \(f\) is homotopic, relative \(A\), to a cellular map.
\end{thm}

\textbf{Reminder.} \enquote{relatively homotopic} means, there is a homotopy \(H\colon X\times [0,1] \to Y\), such that \(f = H(\_, 0) \colon X \to Y\), \(H(\_, 1\colon X \to Y)\) is cellular, \(H(a,t) = f(a)\) for all \(a \in A, t \in [0,1]\).

\begin{example}
    Consider a minimal CW-structure on \(S^n\), i.e. one \(0\)-cell and one \(n\)-cell.
    \(A = X_{-1} = \set{z} = X_0 = \dots = X_{n-1} \subseteq X_n = S^n\).
    Suppose that \(m < n\), give \(S^m\) a minimal CW-structure. Let \(f\colon S^m \to S^n\) be continuous. Take \(z \coloneq f(x)\)

    CAT gives \(f\) is homotpoic to a constant map!

    We can say \(\pi_m(S^n, z) = \set{0}\) for \(m \leq n\)
\end{example}

\begin{proof}[Proof of CAT]
    We start by prooving a special case:
    \begin{thm}{}{}
        Let \(Y = B\cup _{\partial D^n} D^n\). Then for all \(m < n\), every continous map \(f \colon D^m \to Y\) with \(f(\partial D^m) \subseteq B\),is homotopic relative \(\partial D^m\) to a map with image in \(B\).
    \end{thm}
    \begin{proof}
        By induction on \(n\).

        For \(n = 1\), \(m = 0\), \(D^0 = \set{x}\), \(\partial D^0 = \emptyset\).
        \[f\colon \set{x} \to B\cup_{\partial D^1} D^1\]
        is homotpoic to a map with image in \(B\) because \(D^1\) is path connected.
        
        Now let \(n \geq 2\) and assume the special case for all smaller values of \(n\).
        \begin{description}
            \item[Fact 1] For all \(p < n-1\), every continuous map \(S^p \to S^{n-1}\) is homotopic to a constant map.
            \begin{proof}
                By the inductive hypothesis, the composite
                \[D^p \to D^p/S^{p-1} \cong S^{p} \xrightarrow{f} S^{n-1} \cong \set{z} \cup_{\partial D^{n-1}} D^{n-1}\]
                with \(z\coloneq f(\partial D^p)\) is homotopic, relative \(\partial D^p\), to a constant map with value \(\set{z}\).
                Let \(H\colon D^p \times [0,1] \to S^{n-1}\) be such a homotopy. This descends to a  map
                \[\begin{tikzcd}
                    D^p \times [0,1] \ar[r, "H"] \ar[d, "p"] & S^{n-1} \\
                    D^p/\partial D^p \times [0,1] \cong S^p \times [0,1] \ar[ru] & \\
                \end{tikzcd}\]
                which is again continuous.
            \end{proof}

            \item[Fact 2] For \(p < n-1\), every continuous map
            \[h = (h_1, h_2) \colon S^p \to S^{n-1} \times (a,b)\]
            with \(a < b \in \IR\). is homotopic to a constant map.
            \begin{proof}
                Let \(H_1\colon S^p \times [0,1] \to S^{n-1}\) be a homotopy of \(h_1\) to a constant map (Fact 1).
                Let \(H_2\colon S^p \times [0,1] \to (a,b)\) be a linear homotopy from \(h_2\) to some constant map. Then \(H = (H_1, H_2) \colon S^p \times [0,1] \to S^{n-1} \times (a,b)\) is the desired homotopy.
            \end{proof}
            \item[Fact 3] For \(q < n\), every continuous map \(h\colon \partial D^q \to S^{n-1} \times (a,b)\) admits a continuous extension to \(D^q\).
            \begin{proof}
                The map \(\partial D^q \times [0,1] \to D^q\), \((x,t) \mapsto x\cdot t\) is a quotient map. Let \(p = q-1\).
                \(\partial D^q = S^p\), we let \(H\colon \partial D^q \to S^{n-1} \times (a,b)\) be a homotopy from a constant map as in Fact 2.
                \[\begin{tikzcd}
                    {\partial D^q \times [0,1]} \ar[r, "H"] \ar[d, "{(x,t) \mapsto x\cdot t}"'] & {S^{n-1} \times (a,b)} \\
                    D^q \ar[ru, "\overline H"] & \\
                \end{tikzcd}\]
                So there is a continuous map \(\overline{H}\colon D^q \to S^{n-1} \times (a,b)\) with the desired property.
            \end{proof}
        \end{description}
        \textbf{Inductive Step.} Let \(m < n\) and \(f\colon D^m \to Y = B\cup_{\partial D^n} D^n\), such that \(f(\partial D^m) \subseteq B\). We define two open subsets of \(Y\).
        \[U = \set{x \in D^n : \abs{x}< 2/3 }, \quad V = B \cup_{\partial D^n} \set{x \in D^n : \abs{x} > 1/3}\]
        Note that \(U \cap V \cong \partial D^n \times (1/3, 2/3)\). Fact 3 gives: Every continuous map \(\partial D^q \to U\cap V\) admits a continuous extension to \(D^q\) for \(q < n\).

        We replace the pair \((D^m, \partial D^m)\) by the homeomorphic pair \([0,1]^m, \partial([0,1]^m)\). Let
        \[g\colon [0,1]^m \to B\cup_{\partial D^n}D^n = U \cup V, \text{ such that }g(\partial([0,1]^m)) \subseteq B\]
        Then \(g^{-1}(U), g^{-1}(V)\) is an open cover of the compact metric space \([0,1]^m\), so by Lebeques Lemma there is an \(\varepsilon > 0\), such that every \(\varepsilon\)-ball in \([0,1]^m\) is contained in \(g^{-1}(U)\) or in \(g^{-1}(V)\). So we can subdivide \([0,1]^m\) into sufficiently small equally sized and equally spaced subcubes, such that each subcube maps wholly \(U\) or to \(V\) by \(g\).
        \begin{figure}
            \centering
            \includegraphics[width=0.8\linewidth]{pic/excb.png}
            \caption{examples for good/bad cubes.}
            \label{fig:excb}
        \end{figure}
        We need to consider all vertices, edges, squares, \dots, \((m-1)\)- cubes and \(m\)-cubes. Let \(W\) be any such \(p\)-cube. We call \(W\)
        \begin{description}
            \item[Good] if \(g(W) \subseteq V\).
            \item[Bad] if \(g(W) \not\subseteq V\)
        \end{description}
        Note, that
        \begin{itemize}
            \item if \(W\) is bad, then \(g(W) \subseteq U\).
            \item every face of a good cube is good.
            \item every cube contained in \(\partial([0,1]^m)\) is good.
        \end{itemize}
        See figure \ref{fig:excb} for an example.

        Let \(\Gamma\) be the union of all good cubes of all dimension. \(\Gamma \subseteq [0,1]^m\). We define
        \[\begin{array}{rl}
            K^{-1} &= \Gamma = \text{ all good cubes} \\
            K^0 &= K^{-1} \cup \text{ bad \(0\)-cubes} \\
            K^1 &= K^0 \cup \text{ bad \(1\)-cubes} \\
            \vdots & \\
            K^m &= [0,1]^m
        \end{array}\]
        By induction on \(p\) we will construct continuous maps
        \[g_p \colon K^p \to Y = B\cup_{\partial D^n} D^n = U\cup V.\]
        such that:
        \begin{itemize}
            \item \(g_p\rvert_{K^{p-1}} = g_{p-1}\)
            \item if \(W\) is a bad cube, then \(g_p(W) \subseteq U\cap V\).
        \end{itemize}
        Start: \(g_{-1} = g\rvert_\Gamma \colon \Gamma = K^{-1} \to Y\).

        Suppose, that \(g_{-1}, g_0, \dots, g_{p-1}\) have already been constructed.
        
        \textbf{Claim.} If \(W\) is a bad \(p\)-cube, then \(g_{p-1} \subseteq U\cap V\).
        \begin{proof}
            Let \(W'\) be a \(q\)-cube in \(\partial W\), so \(q < p\).
            If \(W'\) is good, then
            \[g_{p-1}(W') = g(W') \subseteq V\]
            But also
            \[g_{p-1}(W') = g(W') \subseteq g(W) \subseteq U\].
            
            If \(W'\) is bad, then \(g_{p-1}(W') \subseteq U\cap W\) by induction hypothesis.
        \end{proof}
        Fact 3 implies, that \(g_{p-1}\rvert_{\partial W}\colon \partial W \to U\cap V \cong \partial D^n \times (1/3,2/3)\) admits a continuous extension to \(W\). We choose such a continuous extension for every bad \(p\)-cube and then define
        \[g_p\colon K^p = K^{p-1} \cup \text{ bad \(p\)-cubes \(\to\) Y} \quad \text{ as }g_{p-1} \cup \text{ chosen extensions}.\]
        This completes the inductive construction of the maps \(g_p \colon K^p \to Y\).
        
        \textbf{Claim.} \(g_m\) and \(g\) are homotopic relative \(\partial [0,1]^m\).
        \begin{proof}
            We show that \(g\) and \(g_m\) are even homotopic relative to \(\Gamma = K^{-1} \supset \partial([0,1]^m)\).

            We write \(C\) for the union of all bad cubes. Then \([0,1]^m = C\cup \Gamma\). Then \(g(C) \subseteq U\) and \(g_m(C) \subseteq U \cap V \subseteq U\). So we can consider the restrictions of both \(g\) and \(g_m\) to \(C\) as continuous maps
            \[g_m\rvert_C, g\rvert_C \colon C \to U \cong R^n\]
            We can use the linear homotopy between \(g_m\) and \(g\). This linear homotopy has the additional property, that it is constant on all points, where \(g\) and \(g_m\) agree. In particular, the homotopy is constant on \(C\cap \Gamma\).
            So the lineare homotopy on \(C\) and the constant homotopy on \(\Gamma\), patch together to a homotopy between \(g_m\) and \(g\), that is moreover constant on \(\Gamma\), hence also constant on \(\partial([0,1]^m)\).
        \end{proof}
        End of the inductive step: We have constructed a homotopy relative to \(\partial([0,1]^m)\) from \(g\) to \(g_m\), which has image in \(V\). \(V\) deformation retracts onto \(B\). Following \(g_m\) with such a deformation retraction, is a relative homotopy from \(g_m\) to a map with image in \(B\).
    \end{proof}

    \begin{thm}{}{}
        Let \((Y,B)\) be a relative CW-complex, and let \(f\colon D^m \to Y\) be a continuous map, such that \(f(\partial D^m) \subseteq B\). Then \(f\) is homotopic,  relative \(\partial D^m\) to a map with image in \(Y_m\).
    \end{thm}
    \begin{proof}
        \textbf{Special case.} \((Y, Y_m)\) is a finite relative CW-complex.
        We argue by induction on the number of relative cells of \((Y, Y_m)\). 
        
        Start: \(Y = Y_m\) is trivial.

        Otherwise, choose a cell of \(Y\) of top dimension n. Then \(m < n\). We choose
        \[Y' = B \cup \text{ all cells of \(Y\) except for the chosen \(n\)-cell}\]
        Then \((Y', B)\) is a relative CW-complex. Hence \((Y', Y_m)\) is a relatively finite CW-complex with one cell less than \((Y, Y_m)\). \(Y = Y' \cup_{\partial D^n} D^n\). By the previous theorem applied to \((Y, Y')\), the map \(f\) is homotopic relative \(\partial D^m\) to a map \(g'\colon D^m \to Y\) with image in \(Y'\). By induction \(g'\) is homotopic relative \(\partial D^m\) to a map \(g''\colon D^m \to Y'\) with image in \(Y_m\). \(g''\) is the desired map.

        \textbf{General case} \(f(D^m)\) is a compact subset of \(Y\), and hence contained in some finite subcomplex  \((\bar Y, B)\) of \((Y,B)\). Apply the special case to \(f\), considered as a map into \(\bar Y\).
    \end{proof}

    \begin{thm}{}{}
        Let \(X\) be obtained from \(A\) by attaching (arbitrarily many) \(n\)-cells. Let \((Y,B)\) be a relative CW-complex. Let \(f\colon X \to Y\) be a continuous map with \(f(A) \subseteq B\). Then \(f\) is homotopic, relative \(A\) to a map with image in \(Y_m\).
    \end{thm}
    \begin{proof}
        We may assume \(X = A \cup_{J\times \partial D^m} J\times D^m\) for some attaching map \(J\times \partial D^m \to A\). For \(j \in J\) we define \(f_j\colon D^m \to Y\) as the composite
        \[\begin{split}
            D^m &\to X = A\cup_{J\times \partial D^m} J\times D^m \xrightarrow{f} Y \\
            x &\mapsto (j,x)
        \end{split}\]
        This satisfies \(f_j(\partial D^m) \subseteq f(A) \subseteq B\).
        The previous special case provides a homotopy \(H_j\colon D^m \times [0,1] \to Y\) relative \(\partial D^m\), from \(f_j\) to a map with image in \(Y_m\).
        We \enquote{glue} the homotopies and the constant homotopy on \(A\) to a homotopy on \(X\), i.e.
        \[\begin{tikzcd}
            {A\times [0,1] \amalg J\times D^n \times [0,1]} \ar[rrr, "{(\const \to f\rvert_A) \amalg \coprod_{j\in J} H_j}"] \ar[d, "{p\times [0,1]}"] &&& Y \\
            {X\times [0,1] = (A\cup_{J\times \partial D^m} J\times D^m) \times [0,1]} \ar[rrru, "\bar H"'] &&& \\
        \end{tikzcd}\]
        where \(p\colon A\amalg J\times D^n \to X\) is the quotient map.
        \(\bar H\) is continuous by the quotient property of \(p\times [0,1]\). \(\bar{H}\) is the desired homotopy. That \(p\times [0,1]\) is a quotient map will be shown later.
    \end{proof}

    \begin{defi}
        A continuous map \(f\colon X\to Y\) is a \emph{quotient map} if it is surjective and \(U\subseteq Y\) is open if and only if \(f^{-1}(U)\) is open
    \end{defi}
    Equivalently: the induced map \(X/\sim_{f} \xrightarrow{\cong} Y\) is a homeomorphism, where \(x \sim_f x' \Lra f(x) = f(x')\).

    In general, if \(f\colon X\to Y\) is a quotient map, then \(f\times Z\colon X\times Z \to Y\times Z\) is continuous and surjective, but not neccessarily a quotient map!

    The next steps will be
    \begin{itemize}
        \item If \(Z\) is locally compact, then \(\times Z\) preserves quoteint maps.
        \item Suppose \(f\colon X\to Y\) is cellular up to level \(m-1\), i.e. \(f(X_k) \subseteq Y_k\) for \(k = -1, 0, 1, \dots, X_{m-1}\), then apply the previous special case to \(f\rvert_{X_m} \colon (X_m, X_{m-1}) \to (Y, Y_{m-1})\) makes \(f\rvert_{X_m}\) homotopic to a cellular map.
        \item Looking at the \emph{Homotopy Extension property}, which some spaces have, allowing to extend a homotopy from a subspace of it to the whole space.
        \item A limit argument to finish the proof.
    \end{itemize} 

    \begin{defi}{}{}
        A space \(X\) is \emph{locally compact}\index{local compactness}, if every neighborhood of any point of \(X\) contains a compact neighborhood of that point.
    \end{defi}
    \begin{lem}{}{}
        Let \(X\) be a space, such that every point has a compact neighborhood. Then \(X\) is locally compact. In particular, compact spaces are locally compact.
    \end{lem}
    \begin{example}
        \(\IR^n\) is locally compact, but not compact.
    \end{example}
    \begin{proof}
        Let \(U\) be a neighborhood of \(x \in X\) in \(X\). Then there is a open set \(U'\) of \(X\) with \(x \in U' \subseteq U \subseteq X\). Let \(K\) be a compact neighborhood of \(x\) in \(X\). Then \(K\setminus U'\) and \(\set{x}\) are disjoint closed subsets of the compact space \(K\). Compact spaces are normal, so there are relatively open subsetes \(W_1\) and \(W_2\) of \(K\), such that \(x \in W_1 \subseteq K\) and \(K\setminus U' \subseteq W_2 \subseteq K\) and \(W_1 \cap W_2 = \emptyset\).
        
        Then \(K\setminus W_2\) is closed in \(K\) an hence compact. Since \(W_1\) is a neighborhood of \(x\) in \(K\) and \(K\) is a neighborhood of \(x \in X\), \(W_1\) is a neighborhood of \(x\) in \(X\). So
        \[x \in W_1 \subseteq K\setminus W_2 \subseteq U \subseteq X.\]
    \end{proof}
    \begin{lem}{Slice lemma}{}
        Let \(X\) and \(Y\) be spaces and \(K\) a compact subset of \(Y\). Let \(x \in X\) and let \(W\) be an open subset of \(X\times Y\), such that \(\set{x} \times K \subseteq W\).Then there is an open subset \(V\) of \(X\), such that \(x \in V\) and \(V\times K\subseteq W\).
    \end{lem}
    This was prooven in GeoTopo.
    \begin{thm}{}{}
        Let \(f\colon X \to Y\) be a quotient map. Then for every locally compact space \(Z\), the map
        \[f\times Z\colon X\times Z \to Y \times Z\]
        is a quotient map.
    \end{thm}
    \begin{proof}
        \(f\times Z\) is continuous and surjective. We must show: Let \(B\subseteq Y \times Z\) such that \(f^{-1}(B)\) is open in \(X\times Z\), then \(B\) is open in \(Y\times Z\).
        
        We consider any point \((y,z) \in B\). We choose some \(x \in X\), such that \(f(x) = y\). Then \((x,z) \in f^{-1}(B)\).
        We define
        \[\begin{split}
            A &\coloneq \set{\bar z \in Z: (y,\bar z) \in B} = \set{\bar z \in Z : (x, \bar z) \in f^{-1}(B)} \\
            &= \text{ preimage of \(B\) under the continuous map } Z\xrightarrow{(y, \_)} Y \times Z
        \end{split}\]
        \(A\) is open in \(Z\). Since \(Z\) is locally compact, there is a compact neighborhood \(K\) of \(z\) inside \(A\).
        \[z \in K \subseteq A \subseteq Z\]
        In particular, \(\set{y} \times K \subseteq B\).
        We define \(U\coloneq \set{\bar y \in Y : \set{\bar y \times K} \subseteq B}\). Then \(y \in U\).

        \textbf{Claim} \(U\) is open in \(Y\).
        \begin{proof}
            Since \(f\colon X \to Y\) is a quotient map, it suffices to show that
            \[f^{-1}(U) = \set{\bar x \in X : \set{\bar x} \times K \subseteq (f\times Z)^{-1}(B)}\]
            is open in X.

            Since \(\bar x \in f^{-1}(U)\) there is an open subset \(V\) of \(\bar x\) in \(X\) with \(V\times K \subseteq (f\times Z)^{-1}(U)\) (Slice Lemma!). Hence \(\bar x \in V \subseteq f^{-1}(U)\) so \(f^{-1}(U)\) is open in \(X\), hence \(U\) is open in \(Y\).
        \end{proof}

        Consider: Given \((y,z) \in B\) we found
        \((y,z) \in U\times K \subseteq B\) with \(U\) open and \(K\) a neighborhood of \(z\).
        So \(B\) is indeed open.
    \end{proof}

    \begin{corollary}
        Let \(X = A \cup_{J\times D^n} J\times D^n\) be obtained from \(A\) by attaching \(n\)-cells. Then for every locally compact space \(Z\), the map \((A\times Z) \amalg (J\times D^n \times Z) \to (A \cup_{J\times \partial D^n} J\times D^n) \times Z = X \times Z\) is a quotient map.
    \end{corollary}
    \begin{proof}
        The map \(f\) is the composite
        \[A\times Z) \amalg (J\times D^n \times Z) \cong (A \amalg J\times D^n) \times Z \to X\times Z\]
        Products commutes with disjoint unions.
    \end{proof}

    \begin{corollary}
        Let \((X,A)\) be a relative CW-complex and \(Z\) a locally compact space. Then for any \(O \subseteq X \times Z\), the following are equivalent:
        \begin{enumerate}
            \item The set \(O\) is open in \(X\times Z\).
            \item For every \(n \geq -1\), \(O\cap (X_n \times Z)\) is open in \(X_n \times Z\)
            \item For every finite subcomplex \((Y,A)\) of \(X\), \(O\cap (Y \times Z)\) is open in \(Y \times Z\).
        \end{enumerate}
    \end{corollary}
    \begin{proof}\leavevmode
        \begin{description}
            \item[1. \(\implies\) 2., 1. \(\implies\) 3.] by subspace topology.
            \item[2. \(\implies\) 1.] We define
            \[\bar X = X_{-1} \amalg X_0 \amalg X_1 \amalg \dots \amalg X_n \amalg \dots.\]
            Let \(\bar f\colon \bar X \to X\) be the inclusion on all \(X_m\). \( \bar f\) is a quotient map by the weak topology.
            By the theorem, \(\bar f\times Z\colon \bar X \times Z \to X \times Z\) is a quotient map.
            Hence also \(\coprod_{n \geq 1} (X_n \times Z) \to X \times Z\) is a quotient map.
            \item[3. \(\implies\) 1.] Recall from the previous class: Let \((X,A)\) be a relative CW-complex, let \(U \subseteq X\), such that
            \begin{itemize}
                \item \(U\cap A\) is closed in \(A\)
                \item \(U\) intersected with the closure of every cell is closed.
            \end{itemize}
            Then \(U\) is closed.
            \begin{proposition}
                Let \((X,A)\) be a relative CW-complex. Then the tautological map
                \[\coprod_{(Y,A) \text{ finite CW-subcomplex of }(X,A)} Y \to X\]
                is a quotient map.
            \end{proposition}
            \begin{proof}
                Every point of \(X\) is either contained in \(A\) or some open cell of \((X,A)\). Since \((A,A)\) is finite, and the closure of every cell is contained in a finite subcomplex, the map is surjective.
                Let \(U\subseteq X\) be such that \(q^{-1}(U)\) is closed. Then \(U\cap Y\) is closed in \(Y\) for every finite subcomplex \((Y,A)\) of \((X,A)\). This includes \((A,A)\), so \(U\cap A\) is closed in \(A\). The closure \(\bar{e_j}\) of a cell \(e_j\) is contained in some finite subcomplex \((Y,A)\), since \(U\cap Y\) is closed in \(Y\), also \(U\cap \bar{e_j}\) is cloesd in \(\bar{e_j}\). Hence \(U\) is cloesd in \(X\).
            \end{proof}
            Let \(O\subseteq X\times Z\) be such that \(O \cap (Y \times Z)\) is open in \(Y \times Z\) for all finite subcomplexes \((Y,A)\) of \(X\). Then \(B = (X\times Z)\setminus O\) has the property that \(B\cap (Y\times Z)\) is closed in \(Y\times Z\) for every finite subcomplex \((Y,A)\) of \((X,A)\). Since \(Z\) is locally compact, product with \(Z\) preserves quotient maps, so
            \[\begin{tikzcd}
                (\coprod_{(Y,A)} Y) \times Z \ar[dr, "q\times Z"] \ar[rr, phantom, "\cong"] && \coprod_{(Y,A)} (Y \times Z) \ar[dl] \\
                & X\times Z & \\
            \end{tikzcd}\]
        \end{description}
    \end{proof}

    \begin{corollary}
        Let \((X,A)\) be a relative CW-complex, and \(Z\) a locally compact space. Let \(f\colon X\times Z \to Y\) be any map. Then the following are equivalent:
        \begin{enumerate}
            \item \(f\) is continuous.
            \item For all \(n \geq -1\), the map \(f\rvert_{X_n \times Z}\colon X_n \times Z \to Y\) is continuous.
        \end{enumerate}
    \end{corollary}
    \begin{proof}
        \(X\times Z\) has the weak topology of the filtration \(\set{X_n \times Z}_{n \geq -1}\) because
        \[\coprod_{n \geq 1} X_n \times Z \to X \times Z\]
        is a quotient map.
    \end{proof}

    \subsubsection{Homotopy extension property}

    \begin{defi}{}{}
        Let \(X\) be a space and \(A\) a subspace of \(X\). Then \((X,A)\) has the \emph{homotopy extension property}\index{homotopy extension property}, if the following holds: let \(f\colon X\to Y\) be a continuous map and let \(H\colon A\times [0,1] \to Y\) be a homotopy starting with \(f\rvert_A\), i.e. for all \(a \in A\), \(H(a,0) = f(a)\). Then there is a homotopy
        \[\bar H \colon X \times [0,1] \to Y\]
        starting with \(f\)and extending \(H\), i.e.
        \begin{itemize}
            \item for all \(x \in X\), \(\bar H(x,0) = f(x)\)
            \item for all \((a,t) \in A\times [0,1], \bar H(a,t) = H(a,t)\).
        \end{itemize}
    \end{defi}

    \begin{lem}{}{}
        A pair \((X,A)\) has the HEP if and only if for every continuous map \(g\colon X\cup_A A \times [0,1] \to Y\), there is a continuous extension to \(X\times [0,1] \to Y\). Here
        \[X \cup_A A \times [0,1] \coloneq (X\amalg A \times [0,1])/\sim\]
        with \(a \sim (a,0)\) for all \(a \in A\).
    \end{lem}

    \textbf{Warning.}
    \[X\cup_A A\times [0,1] \to X \times \set{0} \cup A \times [0,1] \subseteq X \times [0,1]\] \(x \mapsto (x,0), (a,t) \mapsto (a,t)\)
        need not be a homeomorphism.

    \begin{proposition}
        The  ?? \((f, H)\) of a homotopy extension property is equally defined to a continuous map
        \[\]%tikz

        So \((X,A)\) has the HEP iff \(f\cup_A H\) extends continuously to \(X\times [0,1]\).
    \end{proposition}
    \begin{lem}
        Let \(A\) be a closed subset of \(X\). Then the tautological map
        \[\tau\colon X\cup_A A\times [0,1] \to X \times \set{0} \cup A \times [0,1]\] is a homeomorphism.
    \end{lem}
    \begin{proof}
        We know that \(\tau\) is a continuous bijection. We show that \(\tau\) is also a closed map. Let \(B\subseteq X\cup_A A\times [0,1]\) be a closed subset. Let \(p\colon X \amalg A \times [0,1] \to X \cup_A A\times [0,1]\) be the quotient map. Then \(p^{-1}(B) \cap X\) is closedin \(X\), and \(p^{-1}(B) \cap A \times[0,1]\) is closedin \(A\times [0,1]\). Since \(X\times \set{0}\) is closed in \(X\times [0,1]\), \((p^{-1}(B)\cap X) \times \set{0}\) because \(A\) is closed in \(X\), hence \(A\times [0,1]\) is closed in \(X\times [0,1]\).
        So \(\tau(B)\) is the union of two closed subsets in \(X\times [0,1]\), and continuous in \(X\times \set{0} \cup A \times [0,1]\) and have ?? in \(X\times \set{0} \cup A \times [0,1]\).
    \end{proof}
    \begin{corollary}
        Let \(A\) be a closed subspace of \(X\). Then \((X,A)\) has the HEP if and only if the inclusion \(X\times \set{0} \cup A \times [0,1]\) into \(X\times [0,1]\) has a continuous retraction.
    \end{corollary}
    \begin{proof}
        \begin{itemize}
            \item[\(\Ra\)] Apply the HEP to \(f\colon X \to X\times \set{0} \cup A \times [0,1]\) and \(x \mapsto (x,0)\)
            \(H\colon A \times[0,1] \to \)%same space (a,t) \mapsto (a,t)
            So the HEP gives a continuous map \(\bar H\colon X \times [0,1] \to X\times\set 0 \cup A \times [0,1]\). that extends f\& H
            \item[\(\La\)] Let \(\gamma \colon X \times [0,1] \to X \times \set 0 \cup A \times [0,1]\) be a continuous retraction. Let \(f\colon X\to Y\), \(H\colon A \times [0,1] \to Y\) be a homotopy extension problem. Then
            \[X \times [0,1] \to X \times \set 0 \cup A \times [0,1] \to Y\]
            Then \(\bar H \coloneq \)%composite
            is a homotopy extension of \(f\) and \(H\).
        \end{itemize}
    \end{proof}
    \begin{proposition}
        For every \(m \geq 0\), the pair \((\partial D^m, D^m)\) has the HEP. 
    \end{proposition}
    We exhibit a retraction \(r\colon D^m \times [0,1] \to D^m \times \set 0 \cup \partial D^m \times [0,1]\) to the inclusion.
    For \((x,t) \) in \(D^m \times [0,1]\), the line through \((x,t)\) and \((0,2)\) meets \(D^m \times 0 \cup \partial D^m \times [0,1]\) in exactly one point that varies continuously with \((x,t)\), this point defines \(r(x,t)\).

    \begin{proposition}
        Let \(X\) be a space obtained by attaching \(m\)-cells to \(A\). Then \((X,A)\) has the HEP.
    \end{proposition}
    \begin{proof}
        We construct a continuous retraction to \(X\times \set 0 \cup A \times [0,1] \to X\times [0,1]\). We let \(r \colon D^m \times [0,1] \to D^m \times \set 0 \cup \partial D^m \times [0,1]\) be a continuous retraction to the inclusion.
        We define the retraction \(\rho\colon X \times [0,1] \to X \times \set{0} \cup A \times [0,1]\) as follows:
        \[X\times [0,1] = (A \cup_{J\times \partial D^m} J\times D^m) \times [0,1] \la A \times [0,1] \cup_{J\times \partial D^m \times [0,1]} J\times D^m \times [0,1]\]%%tikz
                arrow down
        \(A\times [0,1] \cup X\times \set 0\) = \(A \times [0,1] \cup_{J\times \partial D^m} J\times D^m\) \(\cong\) \(A\times [0,1] \cup_{J\times \partial D^m \times [0,1]} J\times (D^m \times 0 \cup \partial D^m \times [0,1])\)
    \end{proof}

    \begin{thm}{}{}
        Every relative CW-complex has the HEP.
    \end{thm}
    \begin{proof}
        Let \((X,A)\) be a relative CW-complex. We construct by induction continuous retractions \(r_m\colon X_m \times [0,1] \to X_m \times 0 \cup A\times [0,1]\).
        \begin{description}
            \item[\(m = -1\)] Nothing to do.
            \item[\(m \geq 0\)] Suppose \(r_{m-1}\) has alreadey been constructed. We define \(r_m\) as the composite \(X_m \times [0,1] \to X_m \times \set 0 \cup X_{m-1} \times [0,1] \to X_m \times \set 0 \cup (X_{m-1} \times \set 0 \cup A \times [0,1]) = X_m \times \set 0 \cup A \times [0,1]\). First arrow any retraction from previous proposition, second \(\Id \cup r_{m-1}\).
            
            We now define \(r\colon X\times [0,1]\) as the \enquote{union} of the \(r_m\)s, i.e. any \((x,t) \in X\times [0,1]\) is contained in \(X_m \times [0,1]\) for some \(m \geq 0\). We set \(r(x,t) \coloneq r_m(x,t)\). This is independent of \(m\), because \(r_{m+1}\rvert_{X_m\times [0,1]} = r_m\). Then \(r\rvert_{X_m \times [0,1]} = r_m\) is continuous for all \(m \geq 0\). So \(r\) is continuous because \(X \times [0,1]\) has the weak topology wrt \(\set{X_m \times [0,1]}_{m \geq 0}\).
        \end{description}
    \end{proof}
    \textbf{non-example} Let \(X = [-1,0] \cup \set{1/n : n \geq 1}\), \(A = [-1, 0]\). Claim: \((X,A)\) does not have the HEP.

    Let \(f\colon X \to X\) be the identity, \(H\colon A \times [0,1] \to X \) be \(H(a,t) = (1-t) \cdot a - t\) this is contracting \([-1,0]\) onto \[-1\]. Suppose there existed a homotopy \(\bar H \colon X \times [0,1] \to X\) from the identity that extends \(H\). Then \(\bar H\) would need to be constant on each isolated point \(1/n\). By continuity \(\bar H\) would also have to be the identity on the limit point \(0\), but \(H\) is not.

    Remember \ref{thm:CAT}.

    We will inductively construct the following data: for \(m \geq -1\):
    \begin{itemize}
        \item a continuous map \(f_m \colon X \to Y\)
        \item Homotopy \(H_m \colon X \times [0, 1] \to Y\)
    \end{itemize}
    such that \(f_m\) is \enquote{cellular up to level \(m\)}, i.e. \(f_m(X_k) \subseteq Y_k\) for all \(k = -1, 0, \dots, m\).
    \(H_m\) is a hhomotopy from \(f_{m-1}\) to \(f_m\) relative to \(X_{m-1}\).

    We begin with \(f_{-1} = f\). For \(m \geq 0\) suppose the previous data has been constructed. By a previous special case of CAT applied to \((X_m, X_{m-1})\), \((Y, Y_{m-1})\) and \(f_{m-1}\rvert_{X_m} \colon X_m \to Y\) we obtain a homotopy
    \[H\colon X_m \times [0,1 \to Y]\]
    relative \(X_{m-1}\) from \(f_{m-1} \rvert_{X_m}\) to to some map \(H(\_, 1)\colon X_m \to Y\) such that \(H(X_m \times \set 1) \subseteq Y_m\). The HEP for the pair \((X, X_m)\) applied to \(f_{m-1} \colon X \to Y\) and \(H\) yields a homotopy
    \[H_m \colon X \times [0,1] \to Y\]
    form \(f_{m-1}\) that extends \(H\). Then we set \(f_m \coloneq H_m(\_,1)\colon X \to Y\). This has the desired properties.

    If \(X\) was a finite-dimensional CW-complex we would be done.
    We now define a homotopy \(H\colon X \times [0,1] \to Y\) by \enquote{running through the homotopies \(H_m\) faster and faster.}
    \[H(x,t) = \begin{cases}
        H_0(x, 2t) & 0 \leq t \leq 1/2 \\
        H_1(x, 6 \cdot (t - 1/2)) & 1/2 \leq t \leq 2/3 \\
        \vdots & \\
        H_m(x, (m+1)(m+2) \cdot (t - m/(m+1))) & \text{ for } m/(m+1) \leq t \leq (m+1)/(m+2) \\
        H_m(x,1) & \text{ for } t = 1, \;x \in X_m
    \end{cases}\]
    This map is continuous on \(X \times [0,1]\) by the weak topology because it is continuous on \(X_m \times [0,1]\) for all \(m \geq -1\).
\end{proof}

\enquote{The product of two CW-complexes \enquote{is} a CW-complex (often)}

\textbf{Cells multiply}: There is a homeomorphism \(D^m \times D^n \cong D^{m + n}\) that such \((\partial D^m) \times D^n \cup D^m \times (\partial D^n)\) homeomorphic onto \(partial(D^{m+n})\). picture square = circle

Let \(X\) and \(Y\) be CW-complexes. The conadidate CW-structure on \(X\times Y\) is the \emph{product CW-structure} with skeleta \((X \times Y)_n = \bigcup_{k = 0, \dots, n} X_k \times Y_{n-k}\).

\begin{proposition}[CW-recognition theorem]
    Let \(X\) be a Hausdorff space, \(J_k\) a set for all \(k \geq 0\), and \(q\colon \coprod_{k \geq 0} J_k \times D^k \to X\) a continuous map. Suppose that:
    \begin{enumerate}
        \item For every \(n \geq 0\), the restriction of \(q\) to \(J_n \times \inner{D^n}\) is injective, and the ... set of \(X\) is the disjoint union of \(q(J_n\times \inner{D^n})\) for \(n \geq 0\)
        \item For all \(k \geq 0\) and \(j \in J_k\), the set \(q(j\times \partial D^k)\) is contained in a finite union of sets of the form \(q(i \times D^j)\) for some \(j < k\), \(i \in J_j\).
        \item A subset \(A \subseteq X\) is closed in \(X\) if and only if \(A \cap q(j\times D^k)\) is closed in \(q(j \times D^k)\) for all \(k \geq 0, j \in J_k\).
    \end{enumerate}
    Then setting \(X_n \coloneq  \bigcup_{0 \leq k \leq n} q(J_k \times D^k)\) defines a CW-structure on \(X\).
\end{proposition}

\begin{proof}
    Convenient notation: \(e_j^k \coloneq q(j\times \inner{D^k})\) for \(k \geq 0, j \in J_k\) is the \enquote{\(j\)-th open \(k\)-cell}.
    \(\bar{e_j^k} = \text{closure of } e_j^k = q(j \times D^k)\) \enquote{\(j\)-th closed cell}.

    We show by induction on n, that \(X_n\) is closed in \(X\) and \(X_n\) can be obtained from \(X_{n-1}\) by attaching \(n\)-cells indexed by \(J_n\).

    We write \(\alpha J_n\times \partial D^n \to X_{n-1}\) for the restriction of \(q\).
    \[X_{n-1} \amalg J_n \times D^n \to X\]%%tikz
    arrow down P        arrow up f
    \(X_{n-1} \cup_\alpha J\times D^n\)
    arrow up is continuous and injective with image \(X_n\).

    \textbf{Claim.} \(f\) is a closed map.
    Let \(A \subseteq X_{n-1} \cup_\alpha J\times D^n\) be a closed subset. We want to show, that \(f(A)\) is closed in \(X\). We use 3. and check that \(f(A) \cap \bar{e_j^k}\) is closed in \(\bar{e_j^k}\) for all \(k \geq 0\), \(j\in J_k\).
    \begin{description}
        \item[Case 1] \(k < n\). Then \(\bar{e_j^k} \subseteq X_{n-1}\). Because \(A\) is closed, \(p^{-1}(A)\) is closed, so \(A\cap X_{n-1}\), in \(X_{n-1}\) This is closedin \(X\) by induction. So \(f(A) \cap \bar{e_j^k}\) is closed
        \item[Case 2] \(k = n\) \(p^{-1}(A) \cap (j \times D^n)\) is closed in \(j\times D^n\), which is compact. So \(f(A) \cap \bar{e_j^n}\) is the continuous image of a compact set hence compact in \(X\), hence closed in \(X\), and in \(\bar{e_j^n}\).
        \item[Case 3] \(k > n\). Because \(f(A) \subseteq X_n\), \(f(A) \cap \bar{e_j^k} \subseteq q(j\times \partial D^n) \subseteq \text{ finite union of cells of smaller dimension,}\) each of which are closed in the set by induction. So \(f(A) \cap \bar{e_j^k}\) is closed.
    \end{description}
    \(X\) has the weak topology: Let \(A \subseteq X\) be such that \(A \cap X_n\) is closed in \(X_n\) for all \(n \geq 0\). Then \(A \cap \bar{e_j^k}\) is closed in \(\bar{e_j^k}\) for all \(k \geq 0\), \(j \in J_k\) because \(\bar{e_j^k} \subseteq X_k\). By 3. \(A\) is closed in \(X\).
\end{proof}

\textbf{non-example.} \(D^2 = \bigcup_{j \in \partial D^2} \set{j} \cup \inner{D^2}\) is a union of uncountably many open \(0\)-cells, and one \(2\)-cell.

\(q\colon (\partial D^2)_{\text{discret}} \amalg D^2 \to D^2\) the tautological map. This does not define a CW-structure on \(D^2\). The ffiniteness in 2 fails. Because \(\partial D^2\) is not contained in a finite union of cells of dimension \(\leq 1\).

\begin{thm}{}{}
    Let \(X,Y\) be CW-complexes such that \(Y\) is locally compact. Then \((X\times Y)_n \coloneq \bigcup_{k \leq n} X_k \times Y_{n-k}\) defines a CW-structure on \(X \times Y\).

    The \(n\)-cells of this \emph{product CW-structure}\index{product CW-structure} biject with pairs of
    \[\bigcup_{k = 0, \dots, n} (k \text{-cells of } X) \times ((n-k)\text{-cells of Y})\]
\end{thm}
\begin{proof}
    We choose indexing sets and characteristic maps for the given CW-structure on \(X\) and \(Y\). This yields two quotient maps
    \[q\colon \coprod_{k \geq 0} J_k \times D^k \to X \quad q'\colon \coprod_{l \geq 0} J_l' \times D^l \to Y\]
    The product yields a continuous map
    \[\coprod_{k,l \geq 0} J_k \times J_l' \times D^{k+l} \cong (\coprod_{k \geq 0} J_k \times D^n) \times (\coprod_{l \geq 0} J_l' \times D^l) \xrightarrow{q\times q'} X\times Y\]
    The composite satisfies condition 1 and 2 of the previous \enquote{recognition theorem} for CW-structures.

    \textbf{Claim.} \(q\times q'\) is a quotient map.
    \begin{proof}
        \[(\coprod_{k \geq 0} J_k \times D^n) \times (\coprod_{l \geq 0} J'_l \times D^l) \xrightarrow{\Id \times q'} (\coprod_{k \geq 0} J_k \times D^k)\times Y \xrightarrow{q \times Y} X\times Y\]
        first: quoteint maps  because \(\coprod_{k \geq 0} J_k \times D^k\) is disjoint union of compact spaces.
        second: Quoteint map because \(Y\) is locally compact.
    \end{proof}
    Condition 3 of recognitnion theorem: Let \(A \subseteq X\times Y\) be a subset such that \(A \cap e_j^k \times e_j'^l = A \times (\bar{e_j^k} \times \bar{e_{j'}^l})\) is closed in \(\bar{e_j^k} \times \bar{e_{j'}^l}\) for all \(k \geq 0\), \(l \geq 0\), \(j \in J_k\), \(j' \in J_l\).
    Then \((q\times g')^{-1}(A) \cap ((j,j') \cap D^k \times D^l) = (q \times q')^{-1}\rvert_{(j,j') \times D^k \times D^l}(A \cap(\bar{e_j^k} \times \bar{e_{j'}^l}))\) is closed. Since \((q\times q')^{-1}(A)\) is closed and \(q\times q'\) is a quotient map, \(A \) is indeed closed in \(X \times Y\).
\end{proof}




\section{Higher homotopy groups}

\section{singular homology groups}

\end{document}